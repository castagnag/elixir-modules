\documentclass[a4paper,12pt]{article}
\usepackage[T1]{fontenc}
\usepackage{ebgaramond}
\usepackage{lineno}
\usepackage{amsmath}
\author{Aghilas Y. Boussaa}
\sloppy
\newcommand{\kw}[1]{%
  \,\text{\textbf{#1}}\,
}
\newcommand{\pipe}{%
\; | \;
}
\begin{document}
\section{Behaviours}
To specify a behaviour, programmers would use the syntax of
figure~\ref{behaviour-surface}. It maps to the EBNF description given in
figure~\ref{behaviour-formal1}, where $X$ denotes an identifier, $t$ a type and
$\overline{X}$ a list of identifiers. This formal syntax is translated to the
one given in figure~\ref{behaviour-formal2} by the rules given in
figure~\ref{behaviour-formal1to2}. This second syntax separates \texttt{param}
declarations from the others\footnote{The translation to the core language will
show how \texttt{param} declarations differ from the others.} and puts them at
the beginning of the behaviour definition.
\begin{figure}[h]
\begin{verbatim}
defbehaviour M do
  [ $param X
  | $opaque X X1 ... Xn
  | $type X X1 ... Xn = t
  | callback X : t
  ]*
end
\end{verbatim}
\caption{Surface syntax proposal for behaviours}\label{behaviour-surface}
\end{figure}
\begin{figure}[h]
  \begin{align*}D_B^S &::= \kw{param} X \pipe \kw{type} X \overline{X} = T \pipe \kw{opaque} X
  \overline{X} \pipe \kw{callback} X:T \pipe D_B^S ; D_B^S \pipe \epsilon \\
  B^S &::= \kw{defbehaviour} X \kw{do} D_B^S \kw{end}
  \end{align*}
  \caption{A first formal syntax for behaviours}\label{behaviour-formal1}
\end{figure}
\begin{figure}[h]
  \begin{align*}D_B &::= \kw{type} X \overline{X} = T \pipe \kw{opaque} X
  \overline{X} \pipe \kw{callback} X:T \pipe D_B ; D_B \pipe \epsilon \\
  B &::= \kw{param} \overline{X}; D_B
  \end{align*}
  \caption{A second formal syntax for behaviours}\label{behaviour-formal2}
\end{figure}
\begin{figure}[h]
  \caption{Translation rules between the two formal syntaxes}\label{behaviour-formal1to2}
\end{figure}
\section{Modules}
Like with behaviours, we give a surface syntax for modules (figure~\ref{module-surface}), two formal syntaxes (figure~\ref{module-formal1} and figure~\ref{module-formal2}) and the translation rules between them (figure~\ref{module-1to2}).
\begin{figure}[h]
\begin{verbatim}
defmodule M do
  [ $param X
  | $param X=t
  | $type X=t
  | @behaviour B
  | def f(X1, ..., Xn) = e
  ]*
end
\end{verbatim}
\caption{Surface syntax proposal for behaviours}\label{module-surface}
\end{figure}
\begin{figure}[h]
  \begin{align*}
    D_M^S ::=& \kw{param} X \pipe \kw{param} X = t \pipe \kw{type} X \overline{X} = t \pipe \kw{behaviour} X \\
    |& \kw{def} X(\overline{X,}) = E \pipe D_M^S; D_M^S \pipe \epsilon \\
    M^S ::=& \kw{defmodule} X \kw{do} D_M^S \kw{end}
  \end{align*}
  \caption{A first formal syntax for modules}\label{module-formal1}
\end{figure}
\begin{figure}[h]
  \begin{align*}
    D_M ::=&\,\text{\textbf{param}}_X\, X = t \pipe \kw{param} X \pipe \,\kw{type}X \overline{X}=t\pipe \kw{behaviour} X\\
    |& \kw{def} X(\overline{X,}) = E \pipe D_M; D_M \pipe \epsilon \\
    M ::=& D_M
  \end{align*}
  \caption{A second formal syntax for modules}\label{module-formal2}
\end{figure}
\begin{figure}[h]
  \caption{Translation rules between the two formal syntaxes}\label{module-1to2}
\end{figure}
\section{Core language}
We use 1ML's core language augmented with intersection types.
\begin{figure}[h]
  \caption{Syntax of the core language}
\end{figure}
\end{document}
